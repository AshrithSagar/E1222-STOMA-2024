\section*{Question 2}

Let \( X \) be uniform over \( (0,2) \) and let \( Y \) be a discrete random variable taking non-negative integer values.
Suppose \( X, Y \) are independent.
Let \( Z=X+Y \).
Show that \( Z \) is a continuous random variable.

\subsection*{Solution}

We have that \( X \sim U(0,2) \) and \( Y \) is a discrete random variable, taking non-negative integer values.
Since \( X, Y \) are independent, we have that
\begin{equation*}
    f_{XY}(x, y)
    =
    f_{X}(x) f_{Y}(y)
    =
    \frac{1}{2} \mathbf{1}_{(0,2)}(x) f_{Y}(y)
\end{equation*}
where \( f_{Y}(y) \) is the pmf of \( Y \).
Now, we can find the distribution function of \( Z \) as follows
\begin{align*}
    F_{Z}(z)
     & =
    P[Z \leq z]
    =
    P[X+Y \leq z]
    \\ & =
    \int_{-\infty}^{\infty} \int_{-\infty}^{z-x} f_{XY}(x, y) \, dy \, dx
    \\ & =
    \int_{0}^{2} \int_{0}^{z-x} \frac{1}{2} \mathbf{1}_{(0,2)}(x) f_{Y}(y) \, dy \, dx
    \\ & =
    \int_{0}^{2} \frac{1}{2} \mathbf{1}_{(0,2)}(x) \int_{0}^{z-x} f_{Y}(y) \, dy \, dx
    \\ & =
    \int_{0}^{2} \frac{1}{2} \mathbf{1}_{(0,2)}(x) F_{Y}(z-x) \, dx
    \\ & =
    \int_{0}^{2} \frac{1}{2} \mathbf{1}_{(0,2)}(x) \sum_{y=0}^{\lfloor z-x \rfloor} f_{Y}(y) \, dx
    \\ & =
    \frac{1}{2} \int_{0}^{2} \sum_{y=0}^{\lfloor z-x \rfloor} f_{Y}(y) \, dx
    \\ & =
    \frac{1}{2} \sum_{y=0}^{\lfloor z \rfloor} \int_{0}^{2} f_{Y}(y) \, dx
    \\ & =
    \frac{1}{2} \sum_{y=0}^{\lfloor z \rfloor} 1
    =
    \frac{1}{2} \lfloor z \rfloor
\end{align*}

Now, we can find the density function of \( Z \) as follows
\begin{align*}
    f_{Z}(z)
     & =
    \frac{d}{dz} F_{Z}(z)
    =
    \frac{d}{dz} \left( \frac{1}{2} \lfloor z \rfloor \right)
    =
    \frac{1}{2} \delta(z)
\end{align*}

Thus, we have that \( Z \) is a continuous random variable.
