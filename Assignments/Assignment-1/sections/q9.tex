\section*{Question 9}

An interval of length 1 is broken at a point uniformly distributed over \( (0,1) \).
Let \( c \) be a fixed point in \( (0,1) \).
Find the expected length of the subinterval that contains the point \( c \).
Show that this is maximized when \( c=0.5 \).

\subsection*{Solution}

Let the interval be denoted by \( [y, y+1] \).
Let \( X \) be the point at which the interval is broken.
Then, \( X \) is uniformly distributed over \( (0,1) \), i.e., \( X \sim U(0, 1) \).
We have \( y \leq x \leq y+1 \) for some \( y \).
With \( c \) being a fixed point in \( (0,1) \), we have,
\begin{enumerate}
    \item If \( c \leq y \), then the length of the subinterval that contains \( c \) is \( 0 \).
    \item If \( c \geq y+1 \), then the length of the subinterval that contains \( c \) is \( 0 \).
    \item If \( y < c < y+1 \), then we have
          \begin{enumerate}
              \item If \( x < c \), then the length of the subinterval that contains \( c \) is \( y+1-x \).
              \item If \( x > c \), then the length of the subinterval that contains \( c \) is \( x-y \).
          \end{enumerate}
\end{enumerate}
We can write this as
\begin{align*}
    L
     & =
    \begin{cases}
        y+1-x
         &
        \text{if } x < c < y + 1
        \\
        x-y
         &
        \text{if } y < c < x
        \\
        0
         &
        \text{otherwise}
    \end{cases}
\end{align*}
where \( L \) is the length of the subinterval that contains \( c \). \\
We can see that if \( y < -1 \) or \( y > 1 \), then the length of the subinterval that contains \( c \) is \( 0 \).

Since \( X \sim U(0, 1) \), we have \( f_X(x) = 1 \) for \( 0 < x < 1 \).
Thus, we have
\begin{align*}
    E[L]
     & =
    \int_{-1}^{1} \int_{0}^{1} L f_X(x) \, dx \, dy
    \\ & =
    \int_{-1}^{1} \int_{0}^{1} L \, dx \, dy
    \\ & =
    \int_{-1}^{1}
    \left(
    \int_{0}^{c} (y+1-x) \, dx
    +
    \int_{c}^{1} (x-y) \, dx
    \right)
    \, dy
    \\ & =
    \int_{-1}^{1}
    \left(
    \left. (y+1)x - \frac{x^2}{2} \right|_{0}^{c}
    +
    \left. \frac{x^2}{2} - yx \right|_{c}^{1}
    \right)
    \, dy
    \\ & =
    \int_{-1}^{1}
    \left(
    (y+1)c - \frac{c^2}{2}
    +
    \frac{1}{2} - \frac{c^2}{2} - y + yc
    \right)
    \, dy
    \\ & =
    \int_{-1}^{1}
    \left(
    2yc + c - c^2 +\frac{1}{2} - y
    \right)
    \, dy
\end{align*}

\begin{align*}
    \implies
    E[L]
     & =
    2
    \int_{0}^{1}
    \left(
    c - c^2 + \frac{1}{2}
    \right)
    \, dy
    \\ & =
    2
    \left(
    c - c^2 + \frac{1}{2}
    \right)
    \\ & =
    2c - 2c^2 + 1
\end{align*}

Thus, the expected length of the subinterval that contains the point \( c \) is \( 2c - 2c^2 + 1 \).

To find the maximum value of \( E[L] \), we differentiate \( E[L] \) with respect to \( c \) and equate it to \( 0 \).
\begin{align*}
    \frac{dE[L]}{dc}
     & =
    2 - 4c
    \\
    \implies
    2 - 4c
     & =
    0
    \\
    \implies
    c
     & =
    \frac{1}{2}
\end{align*}
