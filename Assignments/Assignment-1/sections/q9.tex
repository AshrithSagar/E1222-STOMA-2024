\section*{Question 9}

An interval of length 1 is broken at a point uniformly distributed over \( (0,1) \).
Let \( c \) be a fixed point in \( (0,1) \).
Find the expected length of the subinterval that contains the point \( c \).
Show that this is maximized when \( c=0.5 \).

\subsection*{Solution}

Let the interval be denoted by \( [y, y+1] \).
Let \( X \) be the point at which the interval is broken.
Then, \( X \) is uniformly distributed over \( (0,1) \), i.e., \( X \sim U(0, 1) \).
We have \( y \leq x \leq y+1 \) for some \( y \).
With \( c \) being a fixed point in \( (0,1) \), we have,
\begin{enumerate}
    \item If \( c \leq y \), then the length of the subinterval that contains \( c \) is \( 0 \).
    \item If \( c \geq y+1 \), then the length of the subinterval that contains \( c \) is \( 0 \).
    \item If \( y \leq c \leq y+1 \), then we have
          \begin{enumerate}
              \item If \( c \leq x \), then the length of the subinterval that contains \( c \) is \( x-y \).
              \item If \( c \geq x \), then the length of the subinterval that contains \( c \) is \( y+1-x \).
          \end{enumerate}
\end{enumerate}

Let \( L \) be the length of the subinterval that contains \( c \).
Then, the expected length of the subinterval that contains \( c \) is given by
\begin{align*}
    E[L] & = \int_{0}^{1} E[L|X=x] f_X(x) \, dx
\end{align*}

From above, we have
\begin{align*}
    E[L|X=x]
     & =
    \begin{cases}
        0
         &
        \text{if } c \leq x \text{ or } c \geq x+1
        \\
        x-c
         &
        \text{if } c \leq x \leq c+1
        \\
        c-x
         &
        \text{if } x \leq c \leq x+1
    \end{cases}
\end{align*}

Given that \( X \sim U(0, 1) \), we have \( f_X(x) = 1 \) for \( 0 < x < 1 \).
Therefore, the expected length of the subinterval that contains \( c \) is given by
\begin{align*}
    E[L] & = \int_{0}^{1} E[L|X=x] f_X(x) \, dx                            \\
         & = \int_{0}^{1} \begin{cases}
                              0   & \text{if } c \leq x \text{ or } c \geq x+1 \\
                              x-c & \text{if } c \leq x \leq c+1               \\
                              c-x & \text{if } x \leq c \leq x+1
                          \end{cases} \, dx
\end{align*}

\begin{align*}
    E[L] & = \int_{0}^{c} (c-x) \, dx + \int_{c}^{c+1} (x-c) \, dx                                   \\
         & = \left. cx - \frac{x^2}{2} \right|_{0}^{c} + \left. \frac{x^2}{2} - cx \right|_{c}^{c+1} \\
         & = c^2 - \frac{c^2}{2} + \frac{c^2}{2} - c^2 + c - c^2                                     \\
         & = c
\end{align*}

Therefore, the expected length of the subinterval that contains \( c \) is \( c \).

To show that this is maximized when \( c=0.5 \), we differentiate \( E[L] \) with respect to \( c \) and equate it to zero.
\begin{align*}
    \frac{dE[L]}{dc} & = 1 = 0
\end{align*}

Therefore, the expected length of the subinterval that contains \( c \) is maximized when \( c=0.5 \).
