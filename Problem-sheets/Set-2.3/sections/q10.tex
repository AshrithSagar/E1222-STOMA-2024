\section*{Question 10}

Consider the following density function (called Beta density)
\begin{equation*}
    f(x)=\frac{\Gamma(a+b)}{\Gamma(a) \Gamma(b)} x^{a-1}(1-x)^{b-1}, 0 \leq x \leq 1
\end{equation*}
where \( \Gamma(\cdot) \) is the gamma function and \( a, b \geq 1 \) are parameters.
Show that this is a density as follows.
By definition of gamma function, we have
\begin{equation*}
    \Gamma(a) \Gamma(b)=\int_{0}^{\infty} x^{a-1} e^{-x} d x \int_{0}^{\infty} y^{b-1} e^{-y} d y
\end{equation*}
First bring the integral over \( y \) inside the integral over \( x \).
Now in the inner integral change the variable from \( y \) to \( t \) using \( t=y+x \).
Now change the order of the \( x \) and \( t \) integrals so that the \( x \) integral becomes the inner integral.
Note that this involves recomputing the limits on the integrals by considering the area over which you are integrating.
Now, in the inner integral change the variable from \( x \) to \( s \) using \( x=t s \).
The final expression you get can then be used to show that the above \( f(x) \) is a density.

\subsection*{Solution}
