\section*{Question 9}

Consider the following density function (called gamma density)
\begin{equation*}
    f(x)=\frac{1}{\Gamma(\alpha)}(\lambda x)^{\alpha-1} \lambda e^{-\lambda x}, \quad x>0
\end{equation*}
where \( \Gamma(\alpha) \) is the so called Gamma function given by
\begin{equation*}
    \Gamma(a)=\int_{0}^{\infty} x^{a-1} e^{-x} d x
\end{equation*}
Note that for the gamma density, \( \alpha, \lambda>0 \) are parameters.
Note that the Gamma function satisfies \( \Gamma(n+1)=n \Gamma(n) \) for all positive integers, \( n \), and \( \Gamma(1)=1 \).
Thus, \( \Gamma(n)=(n-1)! \) for all positive integers, \( n \).

Show that the above \( f \) is a density function. What will this density be if \( \alpha=1 \)?

Suppose \( X \) has this density.
Show that \( E X=\frac{\alpha}{\lambda} \).
(Hint: you do not need to do any integration to find this expectation!)

\subsection*{Solution}
