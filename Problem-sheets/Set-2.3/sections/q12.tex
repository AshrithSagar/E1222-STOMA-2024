\section*{Question 12}

Suppose an experiment can result in one of \( r \) possible outcomes and the \( i^{\text {th }} \) outcome has probability \( p_{i}, i=1,2, \cdots, r \).
(Note that \( \sum_{i=1}^{r} p_{i}=1 \) ).
Suppose we have \( n \) independent repetitions of this experiment.
Argue that the probability that the first outcome occurs \( x_{1} \) times, the second \( x_{2} \) times and so on, is
\begin{equation*}
    \frac{n!}{x_{1}!x_{2}!\cdots x_{r}!} p_{1}^{x_{1}} p_{2}^{x_{2}} \cdots p_{r}^{x_{r}}
\end{equation*}
where \( x_{1}+x_{2}+\cdots+x_{r}=n \).
This is known as the mutinomial distribution.
What would this be if \( n=2 \)?

\subsection*{Solution}
