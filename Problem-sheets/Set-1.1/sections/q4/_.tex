\section*{Question 4}

A fair dice is rolled twice.
What is the probability that the sum of the two numbers is greater than 7 given that the first dice showed 3?

\subsection*{Solution}

Let \( X_i \) be the outcome of the \( i \)-th dice roll.
Then, the sample space is given by
\begin{align*}
    S & = \{ (X_1, X_2) \mid X_1, X_2 \in \{ 1, 2, 3, 4, 5, 6 \} \}
\end{align*}
Let \( A \) be the event that the sum of the two numbers is greater than 7 and \( B \) be the event that the first dice showed 3.
Then, we have
\begin{align*}
    A & = \{ (X_1, X_2) \mid X_1, X_2 \in S, X_1 + X_2 > 7 \} \\
    B & = \{ (X_1, X_2) \mid X_1, X_2 \in S, X_1 = 3 \}
\end{align*}

Therefore, the probability that the sum of the two numbers is greater than 7 given that the first dice showed 3 is given by Bayes' rule as,
\begin{align*}
    P(A \mid B) & = \frac{P(A \cap B)}{P(B)}
\end{align*}
Since the dice is fair, we have
\begin{align*}
    P(A \cap B)
     & = P(X_1 = 3, X_2 > 4)           \\
     & = P(X_1 = 3) \cdot P(X_2 > 4)   \\
     & = \frac{1}{6} \cdot \frac{2}{6}
    = \frac{1}{18}
\end{align*}
Also, we have
\begin{equation*}
    P(B) = P(X_1 = 3) = \frac{1}{6}
\end{equation*}
Therefore, we have
\begin{align*}
    P(A \mid B)
     & = \frac{P(A \cap B)}{P(B)} \\
     & = \frac{1/18}{1/6}
    = \frac{1}{3}
\end{align*}

Therefore, the probability that the sum of the two numbers is greater than 7 given that the first dice showed 3 is \(\frac{1}{3}\).
