\section*{Question 14}

A chord is drawn at random in the unit circle. What is the probability that the length of the chord is greater than the side of the inscribed equilateral triangle?\\
(The side of an equilateral triangle inscribed inside a unit circle is \( \sqrt{3} \)).

The solution to the problem depends on how we `define' what is meant by a `random' chord.
One possibility is as follows.
We can think of choosing a random chord to be same as that of choosing a point inside the circle.
This is because any given point inside the circle can be uniquely corresponded to a chord: join the point to the center of the circle and then draw a line through the point perpendicular to the line joining the point to the center.
The resulting chord is what we can uniquely associate with the point.
Using this idea, Specify what is your \( \Omega \) and calculate the above probability and show that it is 0.25.

Another possibility of modelling a `random chord' is as follows.
We think of choosing a random chord to be same as choosing a point on the circle.
This is because we can fix any arbitrary point on the circle and then define a random chord as one obtained by joining this point to a random point on the circle.
Under this idea, specify your \( \Omega \) and show that the required probability is 1/3.

(This is a celebrated problem that led to many philosophical debates in the eighteenth century. But I hope none of you are confused about there being two `correct' answers!)

\subsection*{Solution}
