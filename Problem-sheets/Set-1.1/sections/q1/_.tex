\section*{Question 1}

Using the axioms of probability derive the following properties.
(Here \(AB\) is intersection of \(A\) and \(B\);\@ \(A^c\) is the complement of \(A\)).

\begin{enumerate}[label = (\alph*)]
    \item \( P(A^c)=1 - P(A) \)
    \item If \( A \subset B \) then \( P(A) \leq P(B) \)
    \item \( P(A\cup B)=P(A)+P(B)-P(AB) \) \\
          (Hint: \( A\cup B = (AB^c)\cup(BA^c)\cup(AB), A = AB^c \cup AB \) and similarly for \( B \)).
\end{enumerate}

\subsection*{Solution}

\subsubsection*{(a) \( P(A^c)=1 - P(A) \)}

\begin{proof}
    The complement of \(A\) is the set of all outcomes that are not in \(A\), and is denoted by \(A^c\).
    We have \(A^c = \Omega \setminus A\), i.e., \(A \cup A^c = \Omega \).
    \(A\) and \(A^c\) are exhaustive events.
    \begin{equation}
        P \left( A \cup A^c \right) = P \left( \Omega \right) = 1
    \end{equation}
    Since \(A\) and \(A^c\) are mutually exclusive and events, we have
    \begin{equation}
        P \left( A \cup A^c \right) = P(A) + P(A^c)
    \end{equation}
    The result follows.
\end{proof}

\subsubsection*{(b) If \( A \subset B \) then \( P(A) \leq P(B) \)}

\begin{proof}
    Since \(A \subset B\), we have \(B = A \cup (B \setminus A)\).
    Since \(A\) and \(B \setminus A\) are mutually exclusive, we have
    \begin{equation}
        P(B) = P(A) + P(B \setminus A) \geq P(A)
    \end{equation}
    and since probabilities are non-negative, the result follows.
\end{proof}

\subsubsection*{(c)}
