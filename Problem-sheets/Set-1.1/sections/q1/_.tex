\section*{Question 1}

Using the axioms of probability derive the following properties.
(Here \(AB\) is intersection of \(A\) and \(B\);\@ \(A^c\) is the complement of \(A\)).

\begin{enumerate}[label = (\alph*)]
    \item \( P(A^c)=1 - P(A) \)
    \item If \( A \subset B \) then \( P(A) \leq P(B) \)
    \item \( P(A\cup B)=P(A)+P(B)-P(AB) \) \\
          (Hint: \( A\cup B = (AB^c)\cup(BA^c)\cup(AB), A = AB^c \cup AB \) and similarly for \( B \)).
\end{enumerate}

\subsection*{Solution}

\subsubsection*{(a) \( P(A^c)=1 - P(A) \)}

\begin{proof}
    We know that \( A \) and \( A^c \) are mutually exclusive events. Therefore, \( A \) and \( A^c \) are disjoint sets.
    \begin{align*}
        A \cap A^c    & = \emptyset \\
        P(A \cap A^c) & = 0
    \end{align*}
    By the axiom of probability, we have
    \begin{align*}
        P(A \cup A^c) & = P(A) + P(A^c) \\
        P(A) + P(A^c) & = 1
    \end{align*}
    Therefore, \( P(A^c) = 1 - P(A) \).
\end{proof}

\subsubsection*{(b) If \( A \subset B \) then \( P(A) \leq P(B) \)}

\subsubsection*{(c)}
