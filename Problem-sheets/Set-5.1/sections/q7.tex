\section*{Question 7}

A professor keeps giving a sequence of exams to the class.
The exams are of three types.
Let \( q_{i} \) denote the probability that the class does well on exam of type \( i \).
It is known that \( q_{1}=0.4, q_{2}=0.6, q_{3}=0.8 \).
If the class does well in the current exam, the next exam is equally likely to be any of the three types.
If the class does badly on the current exam, then the next exam would always be of type 3.
What proportion of exams are of type \( i, i=1,2,3 \)?

\subsection*{Solution}

Let the state space be \( S = \{ 1, 2, 3 \} \), where \( 1 \) denotes the exam of type 1, \( 2 \) denotes the exam of type 2, and \( 3 \) denotes the exam of type 3.

The transition matrix is given by
\begin{align*}
    P
     & =
    \begin{bmatrix}
        0.4/3 & 0.4/3 & (0.4/3 + 1 - 0.4) \\
        0.6/3 & 0.6/3 & (0.6/3 + 1 - 0.6) \\
        0.8/3 & 0.8/3 & (0.8/3 + 1 - 0.8)
    \end{bmatrix}
    =
    \frac{1}{30}
    \begin{bmatrix}
        4 & 4 & 22 \\
        6 & 6 & 18 \\
        8 & 8 & 14
    \end{bmatrix}
    =
    \frac{1}{15}
    \begin{bmatrix}
        2 & 2 & 11 \\
        3 & 3 & 9  \\
        4 & 4 & 7
    \end{bmatrix}
\end{align*}
and the stationary distribution is given by
\begin{align*}
     &
    \pi
    =
    P^\top \pi
    \\
    \implies
     &
    \frac{1}{15}
    \begin{bmatrix}
        -13 & 3   & 4  \\
        2   & -12 & 4  \\
        11  & 9   & -8
    \end{bmatrix}
    \begin{bmatrix}
        \pi_{1} \\
        \pi_{2} \\
        \pi_{3}
    \end{bmatrix}
    =
    \begin{bmatrix}
        0 \\
        0 \\
        0
    \end{bmatrix}
    \\
    \implies
     &
    11 \pi_{1} + 9 \pi_{2}
    =
    8 \pi_{3}
    \implies
    2 \pi_{1} + 9(1 - \pi_{3})
    =
    8 \pi_{3}
    \\
    \implies
     &
    2 \pi_{1} + 9
    =
    17 \pi_{3}
    \\
    \implies
     &
    -39 \pi_{1} + 9 \pi_{2} + 12 \pi_{3}
    =
    11 \pi_{1} + 9 \pi_{2} - 8 \pi_{3}
    =
    0
    \\
    \implies
     &
    20 \pi_{3}
    =
    50 \pi_{1}
    \implies
    \pi_{1}
    =
    \frac{2}{5}
    \pi_{3}
    \\
    \implies
     &
    \frac{81}{5} \pi_{3}
    =
    9
    \implies
    \pi_{3}
    =
    \frac{5}{9}
    \\
    \implies
     &
    \boxed{
        \pi_{1}
        =
        \frac{2}{9}
        ,
        \pi_{2}
        =
        \frac{2}{9}
        ,
        \pi_{3}
        =
        \frac{5}{9}
    }
\end{align*}
